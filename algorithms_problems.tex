\documentclass[%
	11pt,
	a4paper,
	utf8,
	%twocolumn
		]{article}	

\usepackage{style_packages/podvoyskiy_article_extended}


\begin{document}
\title{Задачи по теории алгоритмов и структурам данных}

\author{}

\date{}
\maketitle

\thispagestyle{fancy}

\tableofcontents

\section{Задача two-sums (Easy)}

\url{https://leetcode.com/problems/two-sum/description/} Дана коллекция целых чисел \verb|nums| и целое число \verb|target|. Требуется вернуть коллекцию индексов, сумма элементов которых совпадает со значением \verb|target|.

То есть, если \verb|nums = [2, 7, 11, 15]| и \verb|target=9|, то возвращается \verb|[0, 1]|, так как 

\verb|nums[0] + nums[1] = 9|.

Мое решение с использованием средств Python
\begin{lstlisting}[
style = ironpython,
numbers = none
]
import itertools

def two_sum(self, nums: List[int], target: int) -> List[int]:
	for num_left, num_right in itertools.combinations(nums, 2):
		if num_left + num_right == target:
			if num_left != num_right:
				return [nums.index(num_left), nums.index(num_right)]
			else:
				left_index = nums.index(num_left)
				right_index = nums[left_index + 1:].index(num_left) + left_index + 1
				return [left_index, right_index]
\end{lstlisting}

Это решение с точки зрения потребления памяти оказалось лучше 76,89\% предложенных решений (ничего удивительного, так как используются генераторы), но с точки зрения временных издержек лучше только 36.37\% предложенных решений.

Было еще вот такое решение
\begin{lstlisting}[
style = ironpython,
numbers = none
]
def two_sum(self, nums: List[int], target: int) -> List[int]:
	elem_to_idx = {}
	for idx, elem in enumerate(nums):
		diff = target - elem
		if diff in elem_to_idx:
			return [elem_to_idx[diff], idx]
			
		elem_to_idx[elem] = idx
\end{lstlisting}

Оно лучше по времени (42,88\%), но значительно хуже с точки зрения потребления памяти (5,76\%). Кроме того, на мой взгляд код гораздо труднее для понимания.


% Источники в "Газовой промышленности" нумеруются по мере упоминания 
\begin{thebibliography}{99}\addcontentsline{toc}{section}{Список литературы}
	\bibitem{beazley:python-2010}{\emph{Бизли Д.} Python. Подробный справочник. -- СПб.: Символ-Плюс, 2010. -- 864 с.}
	
	\bibitem{beazley:python_cookbook-2019}{\emph{Бизли Д.} Python. Книга рецептов. -- М.: ДМК Пресс., 2019. -- 648 с.}

    \bibitem{mckinney:pandas-2015}{\emph{Маккинли У.} Python и анализ данных, 2015. -- 482 с.}
	
	\bibitem{ramalho:python-2022}{\emph{Рамальо Л.} Python -- к вершинам мастерства: Лаконичное и эффективное программирование. -- М.: МК Пресс, 2022. -- 898 с.}
	
	\bibitem{heydt:pandas-2019}{\emph{Хейдт М., Груздев А.} Изучаем pandas. -- М.: ДМК Пресс, 2019. -- 682 с.}
	
	\bibitem{hostmann:scala-2013}{\emph{Хостманн К.} Scala для нетерпеливых. -- М.: ДМК Пресс, 2013. -- 408 с.}
\end{thebibliography}

%\listoffigures\addcontentsline{toc}{section}{Список иллюстраций}

%\lstlistoflistings\addcontentsline{toc}{section}{Список листингов}

\end{document}
